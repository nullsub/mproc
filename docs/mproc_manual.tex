\documentclass[a4paper, 12pt]{article}
\pagestyle{plain}
\usepackage{textcomp}
\usepackage[none]{hyphenat}
\title{MPROC Manual and Architecture Description}
\author{Christian Deussen}
\date{2013}
\begin{document}
	\maketitle
	\begin{abstract}
	MPROC is a 16 bit CPU built out of TTL logic chips. The 74HCTxxx logic series is used.
	\end{abstract}
	\tableofcontents
	\newpage
	\section{Overview}
	\subsection {Machine word}
	\begin{itemize}
		\item Always one nibble of opcode bits and the second nibble includes which registers are involved.
		\item The next byte can be an immediate value.
		\item Data-bus is 8 bits wide.
		\item Address-bus is 15 bits wide: 0x0000 to 0x7FFF is the ram, 0x8000 to 0xFFFF is the flash! 
		\item Thus we have 32kb flash and 32kb ram. We can execute both from ram and flash.
		\item The Pointer Register PTR is used for 16 memory access. See SET\_PTR.
		\item PC is 16 bits wide: P\_L:0...7; PC\_H:8...15.
		\item ALU is 8 bits wide.
	\end{itemize}

	\subsection {Stack}
	\begin{itemize}
		\item 32kb Stack FIFO. The stack memory can only be addressed via PUSH and POP. No stack pointer arithmetic is possible because the stack is not connected to the address-bus.
			Thus we have a total of 32kb Ram +32KB Stack = 64kb RAM and 32kb ROM.
		\item The Stack pointer is implemented using 74xx193 binary counter chips. 
	\end{itemize}

	\subsection{I/O}
	\begin{itemize}
		\item I/O can be read and written to with PUSH and POP. 
		\item There are 2 output ports and one input port. All 8 bits wide.
	\end{itemize}

	\subsection{Memory}
	\begin{itemize}
		\item Ram(w24129ak12) and Flash(W29EE011) is connected to the same address Bus. The memory devices have both three-states outputs.
			The bank register(PTR) controls with the highest bit(bit14 on the address bus) whether RAM or the flash will be targeted.
			Writing to the flash will not work since a whole page(256bits) needs to be written at once. Maybe I can upgrade this later with a more decent flash chip,
			which will allow easy writing just like the ram.
	\end{itemize}

	\subsection {Registers}
	\small
		\makebox[\linewidth] {
		\begin{tabular}{|c|c|c|c|}
			\hline
			Register & Used by & Note & Width\\ \hline
			SP & PUSH/POP & Stack Pointer & 16\\ \hline
			PC & JMP[Z/C] & ProgramCounter & 16\\ \hline
			IR & - & Holds current instruction & 16\\ \hline
			LR & SAVE\_LR, RET, PUSH/POP & Stores current return address & 16\\ \hline
			PTR & STR, LDR & Pointer Register & 16\\ \hline
			OutputReg[0..1] & POP & Output ports & 8\\ \hline
			InputReg0] & PUSH & Input ports & 8\\ \hline
			Reg[0..3] & MOV, STR, LDR, PUSH/POP & General Purpose registers & 8\\ \hline
		\end{tabular}
		}
	\newpage
	\section{Instructions}
	\small\begin{center}	
		\subsection{JMP(C/Z)-instructions}
		\begin{itemize}
			\item JMP[Z/C] number adds number to PC; [-128 \textless \space number \textless \space 128]
				This enables indirect Jumps.			
			\item If the JMP has one argument, it is used as an offset instead as an address. Encoded with operand 0x00: reg1, number.\newline
				For example JMP -17 jumps 17 bytes up.
		\end{itemize}
		\subsection{Instruction Set}
		\makebox[\linewidth]{
		\begin{tabular}{|c|c|c|c|}
			\hline
			Nibble 2 & Instruction & Note & Decoder\\ \hline
			0x0 & ADD regA, regB & regA + regB. Result in regA & 0\\ \hline
			0x1 & SUB regA, regB & regA - regB. Result in regA & 0 \\ \hline
			0x2 & NOR regA, regB & regA = !(regA OR regB) & 0 \\ \hline
			0x3 & AND regA, regB & regA = regA AND regB & 0\\ \hline
			0x4 & MOV regA, regB & regA = regB & 0\\ \hline
			0x5 & MOVZ regA, regB & MOV if reg0 is zero &  0\\ \hline
			0x6 & JMP regA, regB & set PC\_H to regA, PC\_L to regB & 0\\ \hline
			0x6 & JMP number & Add sigend number to PC & 0\\ \hline
			0x7 & JMPZ regA, regB & JMP if reg0 is zero & 0\\ \hline
			0x7 & JMPZ number & Add sigend number to PC if reg0 is zero & 0\\ \hline
			0x8 & JMPC regA, regB & JMP if carry is set & 0\\ \hline
			0x8 & JMPC number & Add sigend number to PC if carry is set & 0\\ \hline
			0x9 & STR regA & Store regA where PTR points to & 2\\ \hline
			0xA & LDR regA & Load into regA where PTR points to & 1\\ \hline
			0xB & STR\_I regA & Store regA where PTR points to. Post increment PTR & 2\\ \hline
			0xC & LDR\_I regA & Load into regA where PTR points to. Post increment PTR & 1\\ \hline
			0xD & SET\_PTR regA, regB & Set PTR\_H to regA, PTR\_L to regB & 0\\ \hline
			0xE & PUSH regA & Push regA to the stack & 2\\ \hline
			0xF & POP regA & Pop stack item into regA & 1\\	\hline
			0xEF & RET & Restore LR in PC &   \\ \hline
			0xFF & SAVE\_LR & Save PC + 1 in LR &   \\ \hline
		\end{tabular}	
		}
		\begin{itemize}
			\item regA is a register, regB is a register or a number.\newline
		\end{itemize}
	\end{center}
	\newpage
	\subsection{Argument Decode Table 0}
		The second instruction Nibble defines the operand registers.
		Table 0 is used by MOV, and ALU \\
		JMP(Z/C) uses "reg0, number" to encode "JMP(Z/C) PC\_H, number" thus JMP(Z/C) reg0, number cant be used.\\
		\newline
		\makebox[\linewidth] {
		\begin{tabular}{|c|c|p{6cm}|}
			\hline
			Nibble 1 & Involved Registers \\ \hline
			0x0 & reg0, number \\ \hline
			0x1 & reg0, reg1  \\ \hline
			0x2 & reg0, reg2 \\ \hline
			0x3 & reg0, reg3 \\ \hline
			0x4 & reg1, reg0 \\ \hline
			0x5 & reg1, number \\ \hline
			0x6 & reg1, reg2 \\ \hline
			0x7 & reg1, reg3 \\ \hline
			0x8 & reg2, reg0 \\ \hline
			0x9 & reg2, reg1 \\ \hline
			0xA & reg2, number \\ \hline
			0xB & reg2, reg3 \\ \hline
			0xC & reg3, reg0 \\ \hline
			0xD & reg3, reg1 \\ \hline
			0xE & reg3, reg2 \\ \hline
			0xF & reg3, number \\ \hline
		\end{tabular}
		}
	\newpage
	\subsection{Argument Input Table 1}
	Used by LDR(\_I)and POP
	\begin{center}
		\begin{tabular}{|c|c|}
			\hline
			Nibble 1 & Involved Registers \\ \hline
			0x0 & reg0 \\ \hline
			0x1 & reg1 \\ \hline
			0x2 & reg2 \\ \hline
			0x3 & reg3 \\ \hline
			0x4 & PTR\_L\\ \hline
			0x5 & PTR\_H \\ \hline
			0x6 & LR\_LOW \\ \hline
			0x7 & LR\_HIGH \\ \hline
			0x8 & PC\_LOW  \\ \hline
			0x9 & PC\_HIGH \\ \hline
			0xA & io\_out0 \\ \hline
			0xB & io\_out1 \\ \hline
			0xE & --used-- \\ \hline
			0xF & --used-- \\ \hline
		\end{tabular}
		RET is encoded as POP with 0xE as argument nibble.\\
		SAVE\_LR is encoded as POP with 0F  as argument nibble.\\
	\end{center}	
	\subsection{Argument Output Table 2}
	Used by STR(\_I), PUSH
	\begin{center}
		\begin{tabular}{|c|c|}
			\hline
			Nibble 1 & Involved Registers \\ \hline
			0x0 & number \\ \hline
			0x1 & reg0 \\ \hline
			0x2 & reg1 \\ \hline
			0x3 & reg2 \\ \hline
			0x4 & reg3 \\ \hline
			0x5 & PTR\_L \\ \hline
			0x6 & PTR\_H \\ \hline
			0x7 & LR\_L \\ \hline
			0x8 & LR\_H \\ \hline
			0x9 & PC\_L \\ \hline
			0xA & PC\_H \\ \hline
			0xB & io\_out0 \\ \hline
			0xC & io\_out1 \\ \hline
			0xD & io\_in0 \\ \hline
			0xE & io\_in1 \\ \hline
		\end{tabular}
	\end{center}
	\section{Calling Convention}
	\begin{itemize}
		\item Caller saves his working registers(stack)
		\item Arguments are passed in Reg0 and Reg1. Additional arguments are passed via the stack.
		\item Return Address is saved in LR. If the callee wants to call other functions it has to save LR.
		\item Return values are in Reg0 and Reg1. Additional return values are on the stack.
	\end{itemize}
	\section{Hardware Implementation}
	The 74HCT logic family is used. The design is not limited by the fanout. HCT chips have a typical fanout of \textgreater 50. Propagation delay is 9ns, max frequency is 50MHz.
	PC and SP are implemented using 74xx193 binary counters.
	\subsection{Accessing Memory}
	\begin{center}
		\begin{tabular}{|c|p{6cm}|p{6cm}|}
			\hline
			Step & Read & Write \\ \hline
			1 & Apply address to A-bus, clear Not\_OE and NOT\_CS & Apply address to A-bus, Apply data to D-bus, clear Not\_OE and NOT\_CS \\ \hline
			2 & wait \textgreater\ 6ns  & wait \textgreater\ 6ns \\ \hline
			3 & Read Data from Databus/Write read data back to registers & Set NOT\_CS \\ \hline
			4 & Set NOT\_CS  & \\ \hline
		\end{tabular}
	\end{center}
	\subsection{Fetch and Decode}
		\makebox[\linewidth] {
		\begin{tabular}{|c|c|}
			\hline
			Step & Fetch and Decode \\ \hline
			1 & Connect PC to Mem Addr, IR to DBus \\ \hline
			2 & Do Mem Read and IR write Signals \\ \hline
			3 & Inrement PC. Twice when two byte command.\\ \hline
		\end{tabular}
		}
	\newpage
	\subsection{Execution Steps}
	{\scriptsize
		\makebox[\linewidth]{
		\newline
		\begin{tabular}{|c|p{2cm}|p{3cm}|p{3cm}|p{2cm}|}
			\hline
			Command & 1 & 2 & 3 & 4 \\ \hline
			ALU & Write regA to DBUS. High to low triggered & Latch regA from DBUS for the ALU. Happens at the High to low transition. & Write RegB to DBUS, Latch ALU output at low to high edge of state-signal & Fill Reg\\ \hline
			MOV(Z) & Write regB to DBUS & Fill Regs & &\\ \hline
			JMP(C/Z) & Write RegA to DBUS & Fill PC\_LOW. & Write RegB to DBUS. This is done using the multiplexer to use an input selector as an ouput selector & Fill PC\_HIGH \\ \hline
			JMP(Offset) & Write PC\_L to ALU & Latch PC\_L to ALU input & Write RegB to DBUS, Fill PC\_L with ALU and Inc or Dec PC\_H depending on carry &\\ \hline
			SET\_PTR & Write RegB to DBUS & Fill PTR\_L. & Write RegA to DBUS & Fill PTR\_H. \\ \hline
			SAVE\_PTR & & & & \\ \hline
			RET & & & & \\ \hline
			LDR & & & & \\ \hline
			STR & & & & \\ \hline
			PUSH & Decrement SP & & & \\ \hline
			POP & & & Increment SP & \\ \hline
		\end{tabular}
		}
	}
	\newpage
	\section{Current Problems}
		\subsection{DBUS Selector Encoding}
			SAVE\_LR/RET need custom WRITE\_XXX\_DBUS signals. To be able to encode those for the dbus selector another multiplexer needs to be added.
		\subsection{SAVE\_LR Implementation and RET}
			SAVE\_LR does not know whether the following JMP instructions takes one or two bytes. How to decide whether to save PC+1 with SAVE\_LR or PC+2?
		\subsection{ALU}
			ALU carry\_out signal is changed by NOR and AND instructions. Need a flip flop or something similar.
			Do we need a flip flop for REG1\_ZERO\_MOVZ if reg1 is changed during a MOVZ instruction(Which could in turn have an effect on the STATE\_SIGNAL of the MOV instruction)?
		\subsection{Startup}
			How to initialize the registers when power is turned on?
			RING\_CNTR\_CLR needs to go high. This clears the Ring Counter. Is this signal low active?
			CLR\_ALL\_REGS\_NOT needs to go low for a short period and then for the rest of the time HIGH
		\subsection{Memory Access}
			Read: Can we apply the address, clear\_not\_oe and clear\_not\_cs in one instruction and fill the registers in the following instruction? 2 cycle memory access would be possible with this!
	\section{Toolchain}
		\subsection{Assembler}
			Supports .define  and CALL macro. CALL regA, regB equals to SAVE\_LR; CALL regA, regB
		\subsection{Emulator}
			Does not yet support timing simulation. It just executes the binary
\end{document}
