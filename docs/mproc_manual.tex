\documentclass[a4paper, 12pt]{article}
\pagestyle{plain}
\usepackage{textcomp}
\usepackage[none]{hyphenat}
\title{MPROC Manual and Architecture Description}
\author{Christian Deussen}
\date{2013}
\begin{document}
	\maketitle
	\begin{abstract}
		MPROC is a 16 bit CPU built out of TTL logic chips. The 74HCTxxx logic series is used.
	\end{abstract}
	\tableofcontents
	\newpage
		\section{Overview}
	\subsection {Machine word}
	\begin{itemize}
		\item always one nibble of opcode bits and the second nibble includes which registers are involved
		\item the next byte can be an immediate value.
		\item data-bus is 8 bits wide
		\item address-bus is 15 bits wide: 0x0000 to 0x7FFF is the ram, 0x8000 to 0xFFFF is the flash! 
		\item thus we have 32kb flash and 32kb ram. We can execute both from ram and flash
		\item To adress something larger than 8-bit addresses, the BankRegister(BR) is used for the bits 7..15. Affected instructions: STR and LDA
		\item PC is 16 bits wide: P\_L:0...7; PC\_H:8...15
		\item ALU is 8 bits wide
	\end{itemize}

	\subsection {Stack}
	\begin{itemize}
		\item 32kb Stack FIFO. The stack memory can only be addressed via PUSH and POP. No stack pointer arithmetic is possible because the stack is not connected to the address-bus. Thus we have a total of 32kb Ram +32KB Stack = 64kb RAM and 32kb ROM.
		\item The Stack pointer is implemented using 74xx193 binary counter chips. 
	\end{itemize}

	\subsection{I/O}
	\begin{itemize}
		\item I/O can be written to with the pop/ldr and read from with the push/str instructions. There are 2 Output ports and one input port. Always 8 bits wide.
	\end{itemize}

	\subsection{Memory}
	\begin{itemize}
		\item Ram(w24129ak12) and Flash(W29EE011) is connected to the same address Bus. The memory devices have both three-states outputs. The bank register(BR) controls with the highest bit(bit14 on the address bus) whether RAM or the flash will be targeted. Writing to the flash will not work since a whole page(256bits) needs to be written at once. Maybe I can upgrade this later with a more decent flash chip, which will allow easy writing just like the ram.
	\end{itemize}

	\subsection{JMP(C/Z)-instructions}
	\begin{itemize}
		\item Since the jmp instructions do not support offset but only immediate operands, the compiler needs to make sure that if a target of a jmp is not in the current 256byte block, a FAR\_JMP with both  high and low address are emmitted.
		\item If the JMP has only one argument, it is used as an offset instead as an address. Encoded with operand 0x00: reg1, number.
	\end{itemize}

	\subsection {Registers}
	\small\begin{center}
		\begin{tabular}{|c|c|c|c|}
			\hline
			Register & Used by & Width & Note\\ \hline
			SP & Push, Pop & 16 & Stack Pointer\\ \hline
			PC & JMP*, & 16 & ProgramCounter\\ \hline
			IR & - & 16 & Instruction Register, Stores opcode and operand\\ \hline
			BR & .. . & 16 & BR\\\hline
			OutputReg[0..1] & POP & 8 & Output ports\\ \hline
			InputReg0] & PUSH & 8 & Input ports\\ \hline
			Reg[0..3] & MOV, STR, LDR & 8 & General Purpose registers\\ \hline
		\end{tabular}
	\end{center}
	\newpage

	\newpage
	\section{Instructions}
	\subsection{Instruction Set}
	\small\begin{center}
		\begin{tabular}{|c|c|c|c|}
			\hline
			Low Nibble & Instruction & Note & Decode Table\\ \hline
			0x0 & ADD reg1, reg2 & reg1 + reg2. Result in reg1 & A\\ \hline
			0x1 & SUB reg1, reg2 & reg1 - reg2. Result in reg1 & C \\ \hline
			0x2 & NOR reg1, reg2 & reg1 = !(reg1 OR reg2) & C \\ \hline
			0x3 & AND reg1, reg2 & reg1 = reg1 AND reg2 & A\\ \hline
			0x4 & MOV reg1, reg2 & reg1 = reg2 & A\\ \hline
			0x5 & MOVZ reg1, reg2 & MOV if reg1 is zero &  A\\ \hline
			0x6 & JMP reg1, reg2 & set PC\_L to reg1, PC\_H to reg2 & A\\ \hline
			0x7 & JMPZ reg1, reg2 & JMP if reg1 is zero & A\\ \hline
			0x8 & JMPC reg1, reg2 & JMP if carry is set & A\\ \hline
			0x9 & STR reg1 & Store reg1 where BR points to & C\\ \hline
			0xA & LDR reg1 & Load into reg1 where BR points to & B\\ \hline
			0xB & SET\_BR reg1, reg2 & Set BR\_L to reg1, BR\_H to reg2 & A\\ \hline
			0xC & Unused &  &  \\ \hline
			0xD & COUNT\_BR UP/DOWN & increase(UP) or decrease(down) BR by one & A\\ \hline
			0xE & PUSH reg1 & Push reg1 to the stack & C\\ \hline
			0xF & POP reg1 & Pop stack item into reg1 & B\\	\hline
		\end{tabular}
	\end{center}
	\newpage
	\subsection{Argument Decode Table A}
 The second instruction Nibble defines the operand registers.
Table A is used by MOV, ALU, SET\_BR and ALU
	\begin{center}
		\begin{tabular}{|c|c|p{6cm}|}
			\hline
			Opcode Nibble 2 & Involved Registers & Note \\ \hline
			0x0 & reg1, number &JMP(Z/C) uses this operand byte to encode 
JMP(Z/C) pc\_high, number
thus JMP(Z/C) reg1, number cant be used. \\ \hline
			0x1 & reg2, number & \\ \hline
			0x2 & reg3, number & \\ \hline
			0x3 & reg4, number & \\ \hline
			0x4 & reg1, reg2 & \\ \hline
			0x5 & reg1, reg3 & \\ \hline
			0x6 & reg1, reg4 & \\ \hline
			0x7 & reg1, reg1 & \\ \hline
			0x8 & reg2, reg3 & \\ \hline
			0x9 & reg2, reg4 & \\ \hline
			0xA & reg2, reg1 & \\ \hline
			0xB & reg3, reg2 & \\ \hline
			0xC & reg3, reg4 & \\ \hline
			0xD & reg3, reg1 & \\ \hline
			0xE & reg4, reg2 & \\ \hline
			0xF & reg4, reg3 & \\ \hline
		\end{tabular}
	\end{center}	
	\newpage
	\subsection{Argument Input Table B}
	Used by LDA and POP
	\begin{center}
		\begin{tabular}{|c|c|}
			\hline
			Opcode Nibble 2 & Involved Registers \\ \hline
			0x0 & reg1 \\ \hline
			0x1 & reg2 \\ \hline
			0x2 & reg3 \\ \hline
			0x3 & reg4 \\ \hline
			0x4 & BR\_LOW \\ \hline
			0x5 & BR\_HIGH \\ \hline
			0x6 & IO\_OUTPUT\_REGISTER\_1 \\ \hline
			0x7 & IO\_OUTPUT\_REGISTER\_2 \\ \hline
		\end{tabular}
	\end{center}	
	\subsection{Argument Output Table C}
	Used by STR and PUSH
	\begin{center}
		\begin{tabular}{|c|c|}
			\hline
			Opcode Nibble 2 & Involved Registers \\ \hline
			0x0 & reg1 \\ \hline
			0x1 & reg2 \\ \hline
			0x2 & reg3 \\ \hline
			0x3 & reg4 \\ \hline
			0x4 & number \\ \hline
			0x5 & BR\_LOW \\ \hline
			0x6 & BR\_HIGH \\ \hline
			0x7 & IO\_INPUT\_REGISTER\_1 \\ \hline
		\end{tabular}
	\end{center}
	\section{Calling Convention}
		\begin{itemize}
			\item Caller saves his working registers(For example on the stack)
			\item Arguments are passed in Reg1, Reg2, Reg3. Additional arguments are passed via the stack(Reg4 cannot be used since it is needed for the call)
			\item Return Address is passed through the stack. First the Low byte is pushed, than the High byte
			\item Return values are in Reg1 and Reg2. Additional return values are on the stack (Reg3 and 4 cannot be used since they are needed for the return JMP instruction)
		\end{itemize}
	\section{Hardware Implementation}
	The 74HCT logic family is used.  Write here the propagation delay calculation, Power consumption, and performance charachteristics.
	PC and SP are implemented using 74xx193 binary counters. With that no 16 bit addition needs to be done.
	\subsection{Accessing Memory}
	\begin{center}
		\begin{tabular}{|c|p{6cm}|p{6cm}|}
			\hline
			Step & Read & Write \\ \hline
			1 & Apply address to A-bus, clear Not\_OE and NOT\_CS & Apply address to A-bus, Apply data to D-bus, clear Not\_OE and NOT\_CS \\ \hline
			2 & wait \textgreater\ 6ns  & wait \textgreater\ 6ns \\ \hline
			3 & Read Data from Databus/Write read data back to registers & Set NOT\_CS \\ \hline
			4 & Set NOT\_CS  & \\ \hline
		\end{tabular}
	\end{center}
	\subsection{Fetch and Decode}
	\begin{center}
		\begin{tabular}{|c|c|}
			\hline
			Step & Fetch and Decode \\ \hline
			1 & Connect PC to Mem Addr, IR to DBus \\ \hline
			2 & Do Mem Read and IR write Signals \\ \hline
			3 & Connect PC and 1(through multiplexer ctrl)to ALU, ADD ctrl \\ \hline
			4 & Reg write back \\ \hline
		\end{tabular}
	\end{center}	
	\newpage
	\subsection{Execution Steps}
	\begin{center}
		\begin{tabular}{|c|p{2cm}|p{3cm}|p{3cm}|p{2cm}|c|}
			\hline
			Command & 1 & 2 & 3 & 4 & 5 \\ \hline
			ALU & Write Reg A to DBUS. High to low triggered & Latch Reg A from DBUS for the ALU. Happens at the High to low transition. & Write Reg B to DBUS & At the High to low transition of the state signal the ALU output is latched & Fill Reg\\ \hline
			MOV(Z) & Write Reg to DBUS & Fill Regs & & &\\ \hline
			LDA & & & & & \\ \hline
			STR & & & & & \\ \hline
			JMP(C/Z) & Write Reg A to DBUS & Full PC\_LOW with reg. if Reg1 is 0x00 done& Apply Reg B to DBUS. This is done using the multiplexer to use an input selector as an ouput selector & Fill PC\_HIGH with Reg B & \\ \hline
			COUNT\_BR & Increment/Decrement BR & & & & \\ \hline
			PUSH & Decrement SP & & & & \\ \hline
			POP & & & Increment SP & & \\ \hline
			SET\_BR & Write Reg to DBUS & Fill PC\_LOW with reg & Apply the Table 2 multiplexer that decode table 2 is used & Fill BR with Reg 2 & \\ \hline
			& & & & & \\ \hline
		\end{tabular}
	\end{center}
	\newpage
	\section{Current Problems}
	\subsection{ALU}
		ALU carry\_out signal is changed by NOR and AND instructions. Need a flip flop or something similar.
		Do we need a flip flop for REG1\_ZERO\_MOVZ if reg1 is changed during a MOVZ instruction(Which could in turn have an effect on the STATE\_SIGNAL of the MOV instruction)?
	\subsection{Startup}
		How to initialize the registers when power is turned on?
		RING\_CNTR\_CLR needs to go high. This clears the Ring Counter. Is this signal low active?
		CLR\_ALL\_REGS\_NOT needs to go low for a short period and then for the rest of the time HIGH
	\subsection{Memory Access}
		Read: Can we apply the address, clear\_not\_oe and clear\_not\_cs in one instruction and fill the registers in the following instruction? 2 cycle memory access would be possible with this!
\end{document}

